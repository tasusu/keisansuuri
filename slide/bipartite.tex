\RequirePackage{plautopatch}
\documentclass[14pt,aspectratio=169,xcolor=dvipsnames,table,onlytextwidth,dvipdfmx]{beamer}
\usepackage[utf8]{inputenc}
\usepackage{bxdpx-beamer} % dvipdfmxなので必要

\usetheme{Boadilla}
%Beamerフォント設定
\usefonttheme{professionalfonts}
\usepackage[T1]{fontenc}
\usepackage[deluxe,uplatex]{otf} % 日本語多ウェイト化
\usepackage{mlmodern}  % 太いComputer Modern
% MLmodernのバグを修正: cf. https://tex.stackexchange.com/questions/646333/size-of-integral-symbol-in-section-header-with-mlmodern
\DeclareFontFamily{OMX}{mlmex}{}
\DeclareFontShape{OMX}{mlmex}{m}{n}{%
   <->mlmex10%
   }{}
\usepackage{newtxtext} % 数式以外をTXフォントで上書き
% \usepackage[noalphabet,unicode]{pxchfon}
% \setminchofont{KozMinPro-Regular.otf}% 游明朝Regular
% \setboldminchofont{KozMinPro-Bold.otf}% 游明朝Demibold
% \setgothicfont[0]{BIZ-UDGothicR.ttc}% BIZ UDゴシックR
% \setboldgothicfont[0]{BIZ-UDGothicB.ttc}% BIZ UDゴシックB
\renewcommand{\familydefault}{\sfdefault}  % 英文をサンセリフ体に
\renewcommand{\kanjifamilydefault}{\gtdefault}  % 日本語をゴシック体に
\usefonttheme{structurebold} % タイトル部を太字
\setbeamerfont{alerted text}{series=\bfseries} % Alertを太字
\setbeamerfont{section in toc}{series=\mdseries} % 目次は太字にしない
\setbeamerfont{frametitle}{size=\Large} % フレームタイトル文字サイズ
\setbeamerfont{title}{size=\LARGE} % タイトル文字サイズ
\setbeamerfont{date}{size=\small}  % 日付文字サイズ
\usepackage{bxcoloremoji}

% Babel (日本語の場合のみ・英語の場合は不要)
\uselanguage{japanese}
\languagepath{japanese}
\deftranslation[to=japanese]{Theorem}{定理}
\deftranslation[to=japanese]{Lemma}{補題}
\deftranslation[to=japanese]{Example}{例}
\deftranslation[to=japanese]{Examples}{例}
\deftranslation[to=japanese]{Definition}{定義}
\deftranslation[to=japanese]{Definitions}{定義}
\deftranslation[to=japanese]{Problem}{問題}
\deftranslation[to=japanese]{Solution}{解}
\deftranslation[to=japanese]{Fact}{事実}
\deftranslation[to=japanese]{Proof}{証明}
\def\proofname{証明}
\setbeamertemplate{theorems}[normal font] %日本語の場合,定理内を斜体にしない

%Beamer色設定
\definecolor{UniBlue}{RGB}{0,150,200}
\definecolor{UniGreen}{RGB}{3,175,122}
\definecolor{UniPink}{RGB}{255,128,255}
\definecolor{AlertOrange}{RGB}{255,76,0}
\definecolor{AlmostBlack}{RGB}{38,38,38}
\setbeamercolor{normal text}{fg=AlmostBlack}  % 本文カラー
\setbeamercolor{structure}{fg=UniBlue} % 見出しカラー
\setbeamercolor{block title}{fg=UniBlue!50!black} % ブロック部分タイトルカラー
\setbeamercolor{alerted text}{fg=AlertOrange} % \alert 文字カラー
\setbeamercolor{bibliography entry author}{fg=AlmostBlack} % Bib 文字カラー
\setbeamercolor{bibliography entry note}{fg=AlmostBlack} % Bib 文字カラー
\mode<beamer>{
    \definecolor{BackGroundGray}{RGB}{254,254,254}
    \setbeamercolor{background canvas}{bg=BackGroundGray} % スライドモードのみ背景をわずかにグレーにする
}

%フラットデザイン化
\setbeamertemplate{blocks}[rounded] % Blockの影を消す
\useinnertheme{circles} % 箇条書きをシンプルに
\setbeamertemplate{navigation symbols}{} % ナビゲーションシンボルを消す
\setbeamertemplate{footline}[frame number] % フッターはスライド番号のみ

%タイトルページ
\setbeamertemplate{title page}{%
    \vspace{2.5em}
    {\usebeamerfont{title} \usebeamercolor[fg]{title} \inserttitle \par}
    {\usebeamerfont{subtitle}\usebeamercolor[fg]{subtitle}\insertsubtitle \par}

    \vspace{1.5em}
    \begin{columns}[]
        \begin{column}{.35\linewidth}
        \usebeamerfont{titlegraphic}\inserttitlegraphic\par
        \end{column}
        \begin{column}{.6\linewidth}\setlength\topsep{0pt}
            \begin{flushright}
        \usebeamerfont{author}\insertauthor\par
        \usebeamerfont{institute}\insertinstitute \par
        \vspace{3em}
        \usebeamerfont{date}\insertdate\par
            \end{flushright}
        \end{column}
    \end{columns}
}

%biblatex
% \usepackage[style=ext-authoryear,citestyle=authoryear,natbib=true,giveninits=true,maxbibnames=99,maxcitenames=10,url=false,isbn=false,doi=false,articlein=false]{biblatex}
% \DeclareNameAlias{author}{first-last}
% \addbibresource{shrunk.bib}
% \newcommand{\mkbibbracketscol}[1]{\textcolor{UniBlue}{\scriptsize \mkbibbrackets{#1}}}
% \DeclareCiteCommand{\cite}[\mkbibbracketscol]
%   {\usebibmacro{prenote}}
%   {\usebibmacro{citeindex}%
%    \usebibmacro{cite}}
%   {\multicitedelim}
%   {\usebibmacro{postnote}}
% % \renewcommand{\cite}[1]{\textcolor{UniBlue}{\scriptsize\citep{#1}}}
% \renewcommand*{\finalnamedelim}{\addcomma\addspace} % remove "and" before the last author
% \DeclareDelimFormat{nameyeardelim}{\addspace}
% \setbeamertemplate{bibliography item}{}

% Algorithm系
\usepackage{algorithm}
\usepackage[noend]{algorithmic}
\algsetup{linenosize=\color{fg!50}\footnotesize}
\renewcommand\algorithmicdo{:}
\renewcommand\algorithmicthen{:}
\renewcommand\algorithmicrequire{\textbf{Input:}}
\renewcommand\algorithmicensure{\textbf{Output:}}

% TikZ
\usepackage{tikz}
\usetikzlibrary{positioning,shapes,arrows,quotes,shapes.callouts,calc,graphs,graphs.standard,fit,backgrounds,perspective}
\tikzset{point/.style={circle, fill=AlertOrange, inner sep=1pt, text width=1pt}}
\tikzset{bpoint/.style={circle, fill=black, inner sep=1pt, text width=1pt}}
\tikzset{alt/.code args={<#1>#2#3}{\alt<#1>{\pgfkeysalso{#2}}{\pgfkeysalso{#3}}}}
\tikzset{>=latex}
\tikzset{mynodes/.style={circle,white,fill=UniBlue!90,text width=.8em,inner sep=0pt,text centered,font=\footnotesize}}
\tikzset{myedges/.style={thick}}
\tikzset{mypath/.style={ultra thick, draw=AlertOrange}}
\tikzset{mypath2/.style={ultra thick, dashed, draw=UniGreen}}
\tikzset{mysubset/.style={draw,UniBlue,dotted,thick,fill=UniBlue!10}}
\tikzset{graphs/mygraph/.style={nodes={mynodes}, edges={myedges}, empty nodes}}
%% graph macros
\tikzgraphsset{declare={mybipartite}{%
subgraph I_nm [V={1,2,3,4,5,6}, W={a,b,c,d,e,f}];
1 -- {a,b};
2 -- {b,c,d,e};
3 -- {d,f};
4 -- {e,f};
{5,6} -- f;
}}
\tikzgraphsset{declare={mybipartitedi}{%
subgraph I_nm [V={1,2,3,4,5,6}, W={a,b,c,d,e,f}];
1 -> {a,b};
2 -> {b,c,d,e};
3 -> {d,f};
4 -> {e,f};
{5,6} -> f;
}}
\tikzgraphsset{declare={mybipartite2}{%
subgraph I_nm [V={1,2,3}, W={a,b,c}];
1 -- {a,b}; 2 -- {a,b,c}; 3 -- {a,c};
}} 

% Appendix
\usepackage{appendixnumberbeamer}

% macros
\newcommand{\R}{\mathbb{R}}
\newcommand{\Z}{\mathbb{Z}}
\newcommand{\be}{\mathbf{e}}
\newcommand{\ones}{\mathbf{1}}
% \newcommand{\bp}{\mathbf{p}}
\newcommand{\eps}{\varepsilon}
\newcommand{\defiff}{\overset{\text{def}}{\iff}}
\DeclareMathOperator{\poly}{poly}

\usepackage{mathtools}
\DeclarePairedDelimiter{\abs}{\lvert}{\rvert}
\DeclarePairedDelimiter{\norm}{\lVert}{\rVert}
\DeclarePairedDelimiter{\inprod}{\langle}{\rangle}
\usepackage[no-test-for-array]{nicematrix}
\usepackage{diagbox}

\newcommand{\highlightcap}[3][red]{\tikz[baseline=(x.base)]{\node[rectangle,rounded corners,fill=#1!10](x){#2} node[below=0pt of x, color=#1]{#3};}}
\newcommand{\highlight}[2][red]{\tikz[baseline=(x.base)]{\node[rectangle,rounded corners,fill=#1!10](x){#2};}}
\newcommand{\mycite}[2][\footnotesize]{\textcolor{UniBlue}{#1 [#2]}}

%section
% \AtBeginSection[]{
%     \frame[c]{
%         \centering
%         \begin{tikzpicture}
%             \node[white, circle, fill=UniBlue, text centered, text width=.3\linewidth](bullet) {
%                 {\Huge \bf \insertsectionnumber} \\[4em]
%     };
%     \node[text width=.65\linewidth, right=1em of bullet]{%
%         \usebeamerfont{title}\usebeamercolor[fg]{title}\insertsection\par %
%     };
%         \end{tikzpicture}
%     } %目次スライド
% }

%grid
% \usepackage[colorgrid,gridunit=pt,texcoord]{eso-pic}
\newcommand{\postit}[1]{\tikz[overlay, remember picture]{\node[draw,anchor=north east, align=center, rectangle, fill=yellow!10,xshift=-1pt,yshift=-1pt] at (current page.north east) {#1};}}

\usepackage{tcolorbox}
\newtcolorbox{primal}{colback=red!10!white, colframe=red, title=主問題 (P), left=2pt,right=2pt,top=0pt,bottom=0pt}
\newtcolorbox{dual}{colback=blue!10!white, colframe=blue, title=双対問題 (D), left=2pt,right=2pt,top=0pt,bottom=0pt}

% visible on
\usetikzlibrary{overlay-beamer-styles}

\title{計算数理基礎(二部マッチング)}
\author{担当: \textbf{相馬 輔}}

\begin{document}
\maketitle

\begin{frame}<1>[label=outline]{目次}
\begin{tikzpicture}
    \tikzstyle{block} = [rectangle, text width=\textwidth, rounded corners, minimum height=4em]
    \tikzstyle{line} = [draw, -latex]

    \node[block,alt={<1,3->{fill=cyan!10}{fill=red!10}}] (combopt) {
        \begin{columns}
            \begin{column}{0.5\linewidth}
        \textbf{1. 二部マッチングの最大最小定理}
            \end{column}
            \begin{column}{0.4\linewidth}
                \small
                \structure{・} LPの整数性(復習)\\
                \structure{・} K\H{o}nig--Egerv\'aryの定理 
                % \structure{・} 重み付き二部マッチングの最大最小定理
            \end{column}
        \end{columns}
    };
    \node[block,alt={<1-2,4->{fill=cyan!10}{fill=red!10}}, below=.5em of combopt] (LP) {
        \begin{columns}
            \begin{column}{0.5\linewidth}
        \textbf{2. 重みなし二部マッチング}
            \end{column}
            \begin{column}{0.4\linewidth}
                \small
                \structure{・} 増加道アルゴリズム \\
                \structure{・} K\H{o}nig--Egerv\'aryの定理再訪
            \end{column}
        \end{columns}
    };
    \node[block,alt={<1-3,5->{fill=cyan!10}{fill=red!10}}, below=.5em of LP] (TU) {
        \begin{columns}
            \begin{column}{0.5\linewidth}
        \textbf{3. 重み付き二部マッチング}
            \end{column}
            \begin{column}{0.4\linewidth}
                \small
                \structure{・} 相補性条件 \\
                \structure{・} ハンガリー法 
            \end{column}
        \end{columns}
    };
\end{tikzpicture}%
\end{frame}

%%%%%%%%%%%%%%%%%%%%%%%%%%%%%%%%%%%%%%%%%%%%%
\section{二部マッチングの最大最小定理}
\againframe<2|handout:0>{outline}
%%%%%%%%%%%%%%%%%%%%%%%%%%%%%%%%%%%%%%%%%%%%%
\begin{frame}
    \frametitle{二部マッチング}

    \small
    \begin{columns}[T]
    \begin{column}{.6\textwidth}
        \begin{block}{重み付き二部マッチング問題}
            \structure{入力} $G=(V; E)$: 二部グラフ,枝重み$w_e$ ($e \in E$) \\
            \structure{出力} $w(M) := \sum_{e\in M}w_e$が最大のマッチング$M$
        \end{block}

        \pause
        \begin{block}{\alt<2>{IP}{LP}定式化}
            \setlength{\abovedisplayskip}{0pt}
        \begin{alignat*}{2}
            \text{maximize} & \quad  \sum_{e \in E} w_e x_e \\
            \text{subject to} & \quad \sum_{e \in \delta(i)} x_e \leq 1 & \quad (i \in V) \\
            & \quad \alt<2>{x_e \in \{0,1\}}{} \only<3>{{\color{AlertOrange} x_e \geq 0}} & \quad (e \in E)
        \end{alignat*}
        \end{block}
        \onslide<3->{係数行列の完全単模性により\textbf{常に}0--1のLP最適解をもつ!}
    \end{column}
    \begin{column}{.4\textwidth}
        \centering
        \tikz{
            \graph[nodes={mynodes}, edges={myedges}, grow right=4em]{
            subgraph I_nm [V={1,2,3}, W={4,5}];
            1 -- {4,5};
            2 -- {4,5};
            3 -- {4};
            % matching
            {[edges={mypath}] {1,2} -- {4,5}; }
        };
        }
        \footnotesize
        \begin{alignat*}{3}
            &\text{max}\quad  &&  w^\top x  \\
            &\text{s.t.}\quad  && x_{14} + x_{15}  &&\leq 1 \\
            & && x_{24} + x_{25}  &&\leq 1  \\
            & && x_{34}  &&\leq 1  \\
            & && x_{14} + x_{24} + x_{34} &&\leq 1  \\
            & && x_{15} + x_{25} &&\leq 1  \\
            & && x_{14}, x_{15}, x_{24}, x_{25}, x_{34} &&\in \{0, 1\} 
        \end{alignat*}
    \end{column}
    \end{columns}
\end{frame}

\begin{frame}
    \frametitle{二部マッチングの双対問題}
    \small

    \begin{columns}[T]
    \begin{column}{.45\textwidth}
    \begin{primal}
        \setlength{\abovedisplayskip}{0pt}
        \begin{alignat*}{2}
            \text{max} & \quad  \sum_{e \in E} w_e x_e \\
            \text{s.t.} & \quad \sum_{e \in \delta(i)} x_e \leq 1  \quad (i \in V) \\
            & \quad x \geq 0 
        \end{alignat*}
    \end{primal}
    \end{column}
    \begin{column}{.45\textwidth}
    \begin{dual}
        \setlength{\abovedisplayskip}{0pt}
        \begin{alignat*}{2}
            \text{min} & \quad  \sum_{i \in V} y_i \\
            \text{s.t.} & \quad y_i + y_j \geq w_e \quad (e=ij \in E) \\
            & \quad y \geq 0
        \end{alignat*}
    \end{dual}
    \end{column}
    \end{columns}

    \vfill
    双対問題(D)の係数行列も完全単模.よって,枝重み$w_e$ ($e \in E$)が整数なら,(D)は整数最適解をもつ.
\end{frame}

\begin{frame}
    \frametitle{K\H{o}nig--Egerv\'aryの定理}

    $G = (V^+, V^-; E)$を$V^+, V^-$を頂点集合とする二部グラフとする.

    \begin{definition}
        $S \subseteq V^+, T \subseteq V^-$ が\alert{頂点被覆} \\
        $\defiff$ 任意の枝$e = ij \in E$に対し,$i \in S$または$j \in T$.
    \end{definition}

    \pause\vfill
    先のLPにおいて,$w \equiv 1$の場合を考えると,整数性と強双対定理から以下の定理が得られる.

    \begin{theorem}[K\H{o}nig--Egerv\'ary]
        二部グラフ$G$において,次の最大最小定理が成り立つ.
        \[
            \max_{\text{$M$:マッチング}} |M| = \min_{\text{$(S,T)$: 頂点被覆}} |S| + |T|
        \]
    \end{theorem}
\end{frame}

\begin{frame}
    \frametitle{例}

    \begin{columns}[T]
    \begin{column}{.7\textwidth}
        右の二部グラフにおいて,
        
        \begin{itemize}[<+(1)->]
            \item 最大のマッチングは$|M| = 5$
            \item 最小の頂点被覆は$|S| + |T| = 5$
        \end{itemize}

        \bigskip
        \onslide<+->{
        一般に,二部グラフにおいてマッチング$M$が最大であることを示すには,\[ |M| = |S| + |T| \]を満たす頂点被覆$(S,T)$を提示すれば良い.}
        
        \bigskip
        \onslide<+->{
        そのような頂点被覆は,$M$の最適性の証拠といえる\alert{(良い特徴づけ; good characterization)}}
    \end{column}
    \begin{column}{.3\textwidth}
        \centering
        \tikz{
            \graph[mygraph, grow right=5em]{ mybipartite; };
        \onslide<2->{
            \graph[edges={mypath}] {{(1),(2),(3),(4),(5)} -- {(b),(c),(d),(e),(f)}; };
        }
        \begin{scope}[on background layer]
        \onslide<3->{
            \node[mysubset, fit=(1) (2), "$S$" left ]{} ;
            \node[mysubset, fit=(d) (e) (f), "$T$" right]{} ;
        }
        \end{scope}
        }
    \end{column}
    \end{columns}
\end{frame}

%%%%%%%%%%%%%%%%%%%%%%%%%%%%%%%%%%%%%%%%%%%%%
\section{重みなし二部マッチング}
\againframe<3|handout:0>{outline}
%%%%%%%%%%%%%%%%%%%%%%%%%%%%%%%%%%%%%%%%%%%%%
\begin{frame}
    \frametitle{増加道}

    $M$: $G$のマッチング \\
    \begin{definition}
        パス$P$が($M$に対する)\alert{増加道} $\defiff$
        \begin{enumerate}
            \item $P$において$E \setminus M$と$M$の枝が交互に現れる.
            \item $P$の始点と終点は$M$に接続していない頂点.
        \end{enumerate} 
    \end{definition}
    \begin{center}
        \tikz[baseline=(1.base)]{\graph[mygraph]{
            1 -- 2 --[mypath, "{\footnotesize $\in M$}" {below, AlertOrange}] 3 --["{\footnotesize $\notin M$}" below] 4 --[mypath] 5 -- 6 --[mypath] 7 -- 8 ;
        }} \quad \textcolor{AlertOrange}{$\abs M = 3$}
        \pause
        \\ $\downarrow$ 反転 \\[1em]
        \tikz{\graph[mygraph]{
            1 --[mypath] 2 -- 3 --[mypath] 4 -- 5 --[mypath] 6 -- 7 --[mypath] 8 ;
        }} \quad \textcolor{AlertOrange}{$\abs M = 4$}
    \end{center}

    \vfill\pause
    ∴ $M$が最大マッチング $\implies$ 増加道は存在しない
\end{frame}

\begin{frame}
    \frametitle{増加道}

    \begin{lemma}
        $M$が最大でない $\implies$ 増加道が存在.
    \end{lemma}

    \pause
    \structure{証明} \\
    \newcommand{\M}{\textcolor{AlertOrange}{M}}
    \newcommand{\Mp}{\textcolor{UniGreen}{M'}}
    2つのマッチング$\M$, $\Mp$ ($\abs\M < \abs\Mp$) を取る.
    すると,$\M + \Mp$の連結成分は交互道と長さ偶数のサイクルからなる.
    \begin{center}
        \tikz[]{\graph[mygraph]{1 --[bend left, mypath] 2; 1 --[bend right, mypath2] 2;}}
        \quad
        \tikz[baseline=(2.base)]{\graph[mygraph,clockwise]{subgraph I_n[n=4, radius=.7cm]; 
            {[edges={mypath }] 1 -- 2; 3 -- 4; };
            {[edges={mypath2}] 2 -- 3; 4 -- 1; };
        }}
        \quad
        \begin{tabular}{c}
        \tikz{\graph[mygraph]{
            1 --[mypath] 2 --[mypath2] 3 --[mypath] 4 --[mypath2] 5;
        }} \\
        \tikz{\graph[mygraph]{
            1 --[mypath] 2 --[mypath2] 3 --[mypath] 4;
        }} \\
        \tikz{\graph[mygraph]{
            1 --[mypath2] 2 --[mypath] 3 --[mypath2] 4 --[mypath] 5 --[mypath2] 6;
        }} 
        \end{tabular}
    \end{center}
    いま,$\abs\M < \abs\Mp$より,$\M + \Mp$の連結成分の中に,交互道で始点と終点が$\M$の端点でないものが存在する.これは$\M$に対する増加道. \qed
\end{frame}

\begin{frame}
    \frametitle{増加道アルゴリズム}

    補題より,以下のようなアルゴリズムが考えられる.

    \begin{block}{増加道アルゴリズム}
        \begin{algorithmic}[1]
            \STATE $M \gets \emptyset$とする.
            \WHILE{$M$に対する増加道が存在する}
                \STATE 増加道の1つを求めて$P$とする.
                \STATE $P$に沿って枝を反転して,より大きなマッチング$M$を得る.
            \ENDWHILE
            \RETURN $M$
        \end{algorithmic} 
    \end{block}

    \bigskip
    増加道$P$を$A$時間で求められれば,全体は$O(nA)$時間アルゴリズム.
\end{frame}

\begin{frame}
    \frametitle{有向グラフを用いた増加道探索}

    \small
    \begin{columns}
    \begin{column}{.7\textwidth}

    マッチング$M$に対し,$G$の枝を
    \begin{itemize}
        \item $M$の枝は両向き,
        \item $E\setminus M$の枝は$V^+$から$V^-$向き
    \end{itemize}
    に向き付けた有向グラフを$D_M$とする.
    \onslide<2->{
        また,
    \begin{itemize}
        \item $U^+ :=$ $M$に接続していない$V^+$の頂点集合
        \item $U^- :=$ $M$に接続していない$V^-$の頂点集合
    \end{itemize}
    とする.
    }

    \onslide<3->{
    \begin{block}{}
        増加道 $\longleftrightarrow$ $D_M$における$U^+$から$U^-$への有向パス
    \end{block}
    → $O(m+n)$時間で増加道を求められる \hfill {\footnotesize ($m = |E|$, $n = |V|$)}
    }
    \end{column}
    \begin{column}{.3\textwidth}
        \centering
        \tikz{
            \graph[mygraph, grow right=5em]{
            mybipartitedi;
            % matching
            {[edges={mypath}] {1,2,3,4} <-> {b,e,d,f}; }
        };
        \begin{scope}[on background layer]
        \onslide<2->{
            \node[mysubset, fit=(5) (6), "$U^+$" left ]{} ;
            \node[mysubset, fit=(a), "$U^-$" right]{} ;
            \node[mysubset, fit=(c), "$U^-$" right]{} ;
        }
        \end{scope}
        \node[above=10pt of 1]{$V^+$};
        \node[above=10pt of a]{$V^-$};
        }
    \end{column}
    \end{columns}
\end{frame}

\begin{frame}
    \frametitle{例}

    \begin{columns}[T]
    \begin{column}{.5\textwidth}
        \centering
        \vspace{5pt}

        \tikz{
        \graph[mygraph, grow right=5em]{mybipartite};
            \onslide<3>{\graph[edges={mypath}]{
                (1) -- (b);
            };}
            \onslide<4>{\graph[edges={mypath}]{
                {(1),(2)} -- {(a),(b)};
            };}
            \onslide<5>{\graph[edges={mypath}]{
                {(1),(2),(3),(4)} -- {(a),(b),(f),(e)};
            };}
            \onslide<6>{\graph[edges={mypath}]{
                {(1),(2),(3),(4),(5)} -- {(a),(b),(d),(e),(f)};
            };}
        } \\ $G$
    \end{column}
    \begin{column}{.5\textwidth}
        \centering
        \onslide<2->{
        \tikz{
            \graph[mygraph, grow right=5em]{mybipartitedi};
            \onslide<3>{\graph[edges={mypath}]{
                (1) <-> (b);
            };}
            \onslide<4>{\graph[edges={mypath}]{
                {(1),(2)} <-> {(a),(b)};
            };}
            \onslide<5>{\graph[edges={mypath}]{
                {(1),(2),(3),(4)} <-> {(a),(b),(f),(e)};
            };}
            \onslide<6>{\graph[edges={mypath}]{
                {(1),(2),(3),(4),(5)} <-> {(a),(b),(d),(e),(f)};
            };}
        \begin{scope}[on background layer]
        \onslide<2>{
            \node[mysubset, fit=(1) (2) (3) (4) (5) (6), "$U^+$" left ]{} ;
            \node[mysubset, fit=(a) (b) (c) (d) (e) (f), "$U^-$" right]{} ;
        }
        \onslide<3>{
            \node[mysubset, fit=(2) (3) (4) (5) (6), "$U^+$" left ]{} ;
            \node[mysubset, fit=(a), "$U^-$" right]{} ;
            \node[mysubset, fit=(c) (d) (e) (f), "$U^-$" right]{} ;
        }
        \onslide<4>{
            \node[mysubset, fit=(3) (4) (5) (6), "$U^+$" left ]{} ;
            \node[mysubset, fit=(c) (d) (e) (f), "$U^-$" right]{} ;
        }
        \onslide<5>{
            \node[mysubset, fit=(5) (6), "$U^+$" left ]{} ;
            \node[mysubset, fit=(c) (d), "$U^-$" right]{} ;
        }
        \onslide<6>{
            \node[mysubset, fit=(6), "$U^+$" left ]{} ;
            \node[mysubset, fit=(c), "$U^-$" right]{} ;
        }
        \end{scope}
        } \\ $D_M$ 
        }
    \end{column}
    \end{columns}
    \hfill \onslide<6>{【終了】}
\end{frame}

\begin{frame}
    \frametitle{K\H{o}nig--Egerv\'aryの定理再訪}

    \postit{
        \footnotesize
    $\displaystyle
            \max_{\text{$M$:マッチング}} |M| = \min_{\text{$(S,T)$: 頂点被覆}} |S| + |T|
            $
    }

    \small
    \begin{columns}
    \begin{column}{.7\textwidth}
    \begin{theorem}
        アルゴリズムが停止したときの$D_M$において,$U^+$から到達可能な頂点全体を$X$とする.
        $S := V^+ \setminus X$, $T := V^- \cap X$とすると,$(S,T)$は最小頂点被覆.
    \end{theorem}

    \onslide<2->{
    \structure{(証明)} 全ての枝は右向きに進めるので,$(S,T)$は定義から頂点被覆.
    また,増加道が存在しないので,$X \cap U^- = \emptyset$.
    ゆえに$(S,T)$の全ての頂点は$M$の枝を1本だけ被覆しているので,$|S|+|T|=|M|$.  \qed
    }

    \bigskip
    \onslide<3->{K\H{o}nig--Egerv\'aryの定理の,LPを使わないアルゴリズム的証明が得られた.}
    \end{column}
    \begin{column}{.3\textwidth}
        \centering
        \tikz{
            \graph[mygraph, grow right=5em]{
            mybipartite;

            % matching
            {[edges={mypath}] {1,2,3,4,5} <-> {a,b,d,e,f}; }
        };
        \begin{scope}[on background layer]
            % \node[mysubset, fit=(5) (6) (f), "$X$" right]{} ;
            % \path[draw,AlertOrange,dotted,thick,fill=AlertOrange!10] ($(5.north west)+(-.3,.5)$) -- ($(6.south west)+(-.3,-.3)$) -- ($(f.south east)+(.3,-.3)$) -- ($(f.north east)+(.3,.3)$) -- cycle node[left]{$X$};
            \path[draw,AlertOrange!10,fill=AlertOrange!10,line width=1.0cm, rounded corners] (5.north west) -- (6.south west) -- (f.south west) -- (f.north west) -- cycle node[AlertOrange,left]{$X$};
            \node[mysubset, fit=(6), "$U^+$" left ]{} ;
            \node[mysubset, fit=(c), "$U^-$" right]{} ;
            \node[draw,UniPink,dotted,thick,fill=UniPink!10, fit=(1) (2) (3) (4), "$S$" left ]{} ;
            \node[draw,UniPink,dotted,thick,fill=UniPink!10, fit=(f), "$T$" right]{} ;
        \end{scope}
        }
    \end{column}
    \end{columns}
\end{frame}

\begin{frame}
    \frametitle{重みなし二部マッチング(まとめ)}

    \begin{theorem}
        二部グラフにおいて,
        \begin{itemize}
            \setlength{\itemsep}{1em}
            \item 増加道アルゴリズムは最大マッチング$M$を$O(mn)$時間で求める. \hfill {\footnotesize ($m = |E|$, $n = |V|$)}

            \item 同時に,$|M| = |S| + |T|$を満たす頂点被覆$(S,T)$も求められる.
        \end{itemize}
    \end{theorem}
\end{frame}

%%%%%%%%%%%%%%%%%%%%%%%%%%%%%%%%%%%%%%%%%%%%%
\section{重み付き二部マッチング}
\againframe<4|handout:0>{outline}
%%%%%%%%%%%%%%%%%%%%%%%%%%%%%%%%%%%%%%%%%%%%%
\begin{frame}
    \frametitle{重み付き二部(完全)マッチング}
    \small

    マッチング$M$が\alert{完全} $\defiff$ 全ての頂点が$M$に接続 $\iff$ $2|M| = |V|$

    \begin{block}{重み付き二部完全マッチング問題}
        \structure{入力} $G=(V; E)$: 二部グラフ($|V^+|=|V^-|=n/2$),枝重み$w_e$ ($e \in E$) \\
        \structure{出力} $w(M) := \sum_{e\in M}w_e$が最大の完全マッチング$M$
    \end{block}
    ※ 以下,$G$は少なくとも1つの完全マッチングを持つとする.
    \pause

    \begin{columns}[T]
    \begin{column}{.45\textwidth}
    \begin{primal}
        \setlength{\abovedisplayskip}{0pt}
        \begin{alignat*}{2}
            \text{max} & \quad  \sum_{e \in E} w_e x_e \\
            \text{s.t.} & \quad \sum_{e \in \delta(i)} x_e = 1  \quad (i \in V) \\
            & \quad x \geq 0 
        \end{alignat*}
    \end{primal}
    \end{column}
    \begin{column}{.45\textwidth}
    \begin{dual}
        \setlength{\abovedisplayskip}{0pt}
        \begin{alignat*}{2}
            \text{min} & \quad  \sum_{i \in V} y_i =: y(V) \\
            \text{s.t.} & \quad y_i + y_j \geq w_e \quad (e=ij \in E) 
        \end{alignat*}
    \end{dual}
    \end{column}
    \end{columns}

\end{frame}

\begin{frame}
    \frametitle{相補性条件}

    \begin{columns}[T]
    \begin{column}{.6\textwidth}
        \begin{lemma}[相補性条件]
        $M \subseteq E$, $y \in \R^V$が最適解 
        $\iff$
        \begin{enumerate}
            \item $M$は完全マッチング
            \item $y$は(D)の実行可能解
            \item $y_i + y_j = w_e$ \quad ($e = ij \in M$)
        \end{enumerate}
        \end{lemma}
        
        \onslide<2->{
        以下,
                \begin{itemize}
                    \item (D)の実行可能解を\alert{$w$被覆}
                    \item ③を満たす枝を\alert{タイトな枝}
                \end{itemize}
                という.
        }
    \end{column}
    \begin{column}{.35\textwidth}
        \footnotesize
        \begin{primal}
        \begin{alignat*}{2}
            \text{max} & \quad  \sum_{e \in E} w_e x_e \\
            \text{s.t.} & \quad \sum_{e \in \delta(i)} x_e = 1  \quad (i \in V) \\
            & \quad x \geq 0 
        \end{alignat*}
    \end{primal}
    \begin{dual}
        \setlength{\abovedisplayskip}{0pt}
        \begin{alignat*}{2}
            \text{min} & \quad  \sum_{i \in V} y_i =: y(V) \\
            \text{s.t.} & \quad y_i + y_j \geq w_e \quad (e \in E) 
        \end{alignat*}
    \end{dual}
    \end{column}
    \end{columns}

\end{frame}

\begin{frame}
    \frametitle{ハンガリー法}

    \small
    ②③を満たすマッチング$M$と$w$被覆$y$を保持し,①を満たすまで更新するアルゴリズム.

    \begin{definition}
        $w$被覆$y$に対し,$G$の部分グラフ$G_y = (V^+, V^-, E_y)$を
        \[
            E_y = \{e = ij \in E : y_i + y_j = w_e \} 
        \]
        と定める.つまり,$G_y$はタイトな枝のみ残したグラフ.
    \end{definition}

    \begin{lemma}
        $G_y$に完全マッチングが存在すれば,それは最大重みマッチング.
    \end{lemma}
    \structure{証明} 相補性条件!\qed

\end{frame}

\begin{frame}
    \frametitle{ハンガリー法}

    \small
    \begin{columns}[T]
    \begin{column}{.7\textwidth}
    $G_y$に完全マッチングが存在しなかったら?\\
    $\longrightarrow$ $y$を更新し,(D)の目的関数値を改善できる.
    
    \pause
    \begin{lemma}
        $D_M(y)$を$G_y$に対する重みなしマッチングで使う有向グラフとし,$D_M(y)$における$U^+$から到達可能な頂点全体を$X$とする.
            $\eps := \min\{ y_i + y_j - w_e : e = ij \in E, i \in V^+ \cap X, j \in V^- \setminus X \}$ とする.
            $y'$を
            \setlength{\abovedisplayskip}{0pt}
        \[ 
        y'_i =
        \begin{cases}
            y_i - \eps & (i \in V^+ \cap X) \\
            y_i + \eps & (i \in V^- \cap X) \\
            y_i        & (\text{それ以外})
        \end{cases}
        \]
    とすると,$y'$は$w$被覆で,$y'(V) < y(V)$を満たす.
            % \item $\eps = +\infty$ $\implies$ 完全マッチングは存在しない.
    \end{lemma}
    \end{column}
    \begin{column}{.3\textwidth}
    \tikz{\graph[mygraph,simple,grow right=5em, branch down=3em]{
        {[edges=gray!20] mybipartite2;};
    1 -> b, 2 -> {a,b}, 3 -> a;
    {[edges=mypath] {(1),(2)} <-> {(b),(a)}};
    };
    \foreach \y\i in {4/1,2/2,3/3}
        \node[UniGreen,left=1pt of \i]{$\y$};
    \foreach \y\i in {2/a,0/b,0/c}
        \node[UniGreen,right=1pt of \i]{$\y$};
    \begin{scope}[on background layer]
        \path[draw,AlertOrange!10,fill=AlertOrange!10,line width=1.0cm, rounded corners, visible on=<.->] (1.north west) -- (3.south west) -- (b.south east) -- (a.north east) -- cycle node[AlertOrange,left]{$X$};
    \end{scope}
        \foreach \i in {1,2,3}
            \node[UniGreen,above] at (\i) {$-\eps$};
        \foreach \i in {a,b}
            \node[UniGreen,above] at (\i) {$+\eps$};
    } 
    \end{column}
    \end{columns}
    % $\longrightarrow$ $G_y$の最小頂点被覆$(S, T)$は$|S| + |T| < n/2$を満たす.
    
    % \pause
    % \bigskip

    % \pause
    % \bigskip
    % と,$|V^+ \setminus X| + |V^- \cap X| < n/2 \iff |V^+ \cap X| - |V^- \cap X| > 0$.
   

\end{frame}

\begin{frame}
    \frametitle{補題の証明}
    \postit{
        \footnotesize
        $\eps := \min\{ y_i + y_j - w_e : e = ij \in E, i \in V^+ \cap X, j \in V^- \setminus X \}$ \\
        \footnotesize
            $
        y'_i =
        \begin{cases}
            y_i - \eps & (i \in V^+ \cap X) \\
            y_i + \eps & (i \in V^- \setminus X) \\
            y_i        & (\text{それ以外})
        \end{cases}
        $
    }

    \small
    % \structure{Claim} 枝$e = ij \in E$ ($i \in V^+ \cap X$, $j \in V^- \setminus X$)はタイトではない.
    
    \structure{$y'$が$w$被覆であること}
    \begin{itemize}
        \item 制約$y_i + y_j \geq w_e$ ($e = ij \in E$)が更新後に破られる可能性があるのは,$i \in V^+ \cap X$, $j \in V^- \setminus X$の場合のみ.
        \item $\eps$の定義より実行可能性は保たれる.
    \end{itemize}

    \bigskip\pause
    \structure{$y'(V) < y(V)$であること}
    \begin{itemize}
        \item $y'(V) = y(V) - \eps(|V^+ \cap X| - |V^- \cap X|)$.
        
        \item いま,$G_y$に完全マッチングが存在しないので,$|V^+ \setminus X| + |V^- \cap X| < n/2$. \\
        $\longrightarrow |V^+ \cap X| - |V^- \cap X| > 0$.
        
        \item よって,$\eps > 0$を示せば良い.$X$は$G_y$の到達可能集合なので,任意の枝$e = ij \in E$ s.t. $i \in V^+ \cap X$, $j \in V^- \setminus X$ はタイトでない.
        ゆえに$y_i + y_j - w_e > 0$が成り立つ. \qed
    \end{itemize}
    
\end{frame}

\begin{frame}
    \frametitle{ハンガリー法}
    
    \small
    \begin{block}{}
        \begin{algorithmic}[1]
            \STATE $y_i := \max_{e \in \delta(i)} w_e$ ($i \in V^+$), $y_j := 0$ ($j \in V^-$)とする.
            \WHILE{\textbf{True}}
                \STATE $G_y$の(重みなし)最大マッチング$M$を求める.
                \IF{$M$が完全マッチング}
                    \RETURN $M$ 
                \ELSE
                    \STATE $X$を$D_M(y)$における$U^+$から到達可能な頂点全体とする.
                    \STATE $\eps := \min\{ y_i + y_j - w_{ij} : i \in V^+ \cap X, j \in V^- \setminus X \}$
                    \STATE $y_i \gets y_i - \eps$ ($i \in V^+ \cap X$), $y_j \gets y_j + \eps$ ($j \in V^- \cap X$)
                \ENDIF
            \ENDWHILE
            \RETURN $M$
        \end{algorithmic} 
    \end{block}
\end{frame}

\begin{frame}
    \frametitle{例} 
    \small
    \begin{columns}
    \begin{column}{.20\textwidth}<+->
        \centering
        \tikz{\graph[mygraph,grow right=5em, branch down=3em]        {mybipartite2; 
    1 --["{$5$}" above] a;
    1 --["{$4$}" {above, pos=.3}] b;
    2 --["{$4$}" {above, pos=.1}] a;
    2 --["{$2$}" {below, pos=.9}] b;
    2 --["{$1$}" {above, pos=.99}] c;
    3 --["{$5$}" {above, pos=.02}] a;
    3 --["{$2$}" below] c;
        };}
        \\ $G$
    \end{column}
    \begin{column}{.25\textwidth}<+->
        \centering
        \tikz{\graph[mygraph,simple,grow right=5em, branch down=3em]{ 
        {[edges=gray!20]
            mybipartite2;
        };
        {1,2,3} -> a;
        };
        \foreach \y\i in {5/1,4/2,5/3}
            \node[UniGreen,left=1pt of \i]{$\y$};
        \foreach \y\i in {0/a,0/b,0/c}
            \node[UniGreen,right=1pt of \i]{$\y$};
        \onslide<+->{\graph[edges=mypath]{(1) <-> (a)};}
        \begin{scope}[on background layer]
            \path[draw,AlertOrange!10,fill=AlertOrange!10,line width=1.0cm, rounded corners,visible on=<+->] (1.north west) -- (3.south west) -- (a.south east) -- (a.north east) -- cycle node[AlertOrange,left]{$X$};
        \end{scope}
        \onslide<+->{
            \foreach \i in {1,2,3}
                \node[UniGreen,above] at (\i) {$-\eps$};
            \node[UniGreen,above] at (a) {$+\eps$};
        }
        } $y(V)=14$, \onslide<.->{$\eps=1$}
    \end{column}
    \begin{column}{.25\textwidth}<+->
        \centering
        \tikz{\graph[mygraph,simple,grow right=5em, branch down=3em]{
            {[edges=gray!20] mybipartite2;};
        1 -> {a,b}, 2 -> a; 3 -> a;
        };
        \onslide<.>{
            \graph[edges=mypath]{(1) <-> (a);};
        }
        \foreach \y\i in {4/1,3/2,4/3}
            \node[UniGreen,left=1pt of \i]{$\y$};
        \foreach \y\i in {1/a,0/b,0/c}
            \node[UniGreen,right=1pt of \i]{$\y$};
        \onslide<+->{
        \graph[edges=mypath]{{(1),(2)} <-> {(b),(a)}};
        \begin{scope}[on background layer]
            \path[draw,AlertOrange!10,fill=AlertOrange!10,line width=1.0cm, rounded corners,visible on=<.->] (2.north west) -- (3.south west) -- (a.south east) -- (a.north east) -- cycle node[AlertOrange,left]{$X$};
        \end{scope}
            \foreach \i in {2,3}
                \node[UniGreen,above] at (\i) {$-\eps$};
            \node[UniGreen,above] at (a) {$+\eps$};
        }
        } $y(V)=12$, \onslide<.->{$\eps=1$}
    \end{column}
    \begin{column}{.25\textwidth}<+->
        \centering
        \tikz{\graph[mygraph,simple,grow right=5em, branch down=3em]{
            {[edges=gray!20] mybipartite2;};
        1 -> b, 2 -> {a,b}, 3 -> a;
        {[edges=mypath] {(1),(2)} <-> {(b),(a)}};
        };
        \foreach \y\i in {4/1,2/2,3/3}
            \node[UniGreen,left=1pt of \i]{$\y$};
        \foreach \y\i in {2/a,0/b,0/c}
            \node[UniGreen,right=1pt of \i]{$\y$};
        \onslide<+->{
        \begin{scope}[on background layer]
            \path[draw,AlertOrange!10,fill=AlertOrange!10,line width=1.0cm, rounded corners, visible on=<.->] (1.north west) -- (3.south west) -- (b.south east) -- (a.north east) -- cycle node[AlertOrange,left]{$X$};
        \end{scope}
            \foreach \i in {1,2,3}
                \node[UniGreen,above] at (\i) {$-\eps$};
            \foreach \i in {a,b}
                \node[UniGreen,above] at (\i) {$+\eps$};
        }
        } $y(V) = 11$, \onslide<.->{$\eps=1$}
    \end{column}
    \end{columns}

    \begin{columns}
    \begin{column}{.35\textwidth}<+->
        \tikz{\graph[mygraph,simple,grow right=5em, branch down=3em]{
            {[edges=gray!20] mybipartite2;};
        1 -> b, 2 -> {a,b,c}, 3 -> {a,c};
        {[edges=mypath] {1,2} <-> {b,a};};
        };
        \foreach \y\i in {3/1,1/2,2/3}
            \node[UniGreen,left=1pt of \i]{$\y$};
        \foreach \y\i in {3/a,1/b,0/c}
            \node[UniGreen,right=1pt of \i]{$\y$};
        \onslide<+->{
        \graph[edges=mypath]{{(1),(2),(3)} <-> {(b),(a),(c)}};
        }
        } $y(V) = 10$\onslide<.->{ $=w(M)$ 【終了】}
    \end{column}
    \begin{column}{.65\textwidth}
    
    \end{column}
    \end{columns}
    

\end{frame}

\begin{frame}{ハンガリー法の解析}

    \small
    \begin{theorem}
        ハンガリー法は$O(mn^3)$時間で最大重み完全マッチングと最小$w$被覆を求める.
        \hfill {\footnotesize ($m = |E|$, $n = |V|$)}
    \end{theorem}

    \pause
    \begin{itemize}[<+->]
        \item アルゴリズムの実行中に$|M|$は減らない.
        \begin{itemize}
            \item ($\because$) $y$が更新されて$y'$になったとき,$G_y$の最大マッチング$M$は$G_{y'}$でもマッチング.
        \end{itemize}
        \item よって,$M$は高々$n/2$回更新される.
        \item また,$|M|$が増えず$y$だけが更新されるとき,$\eps$の定義のminを達成する枝により,到達可能集合$X$が真に拡大する.\\
        $\longrightarrow$ 高々$n$回の$y$の更新後に,$|M|$は増える.

        \item ∴全体では$O(n^2)$回の反復.
    \end{itemize}

    \vfill
    \onslide<4->
\end{frame}

\begin{frame}
    \frametitle{まとめ}

    \begin{itemize}
        \setlength{\itemsep}{1em}
        \item LPを用いた組合せ最適化の基本的な考え方(整数多面体,完全単模行列)

        \item LPの整数性を用いて,二部マッチングの最大最小定理(K\H{o}nig--Egerv\'aryの定理)を証明した

        \item LPを用いない組合せ的アルゴリズム
        \begin{itemize}
            \item 重みなし二部マッチングの増加道アルゴリズム
            \item 重み付き二部完全マッチングのハンガリー法
        \end{itemize}
    \end{itemize}
    
\end{frame}

\end{document}
