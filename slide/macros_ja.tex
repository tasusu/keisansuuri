\usepackage[utf8]{inputenc}
\usepackage{bxdpx-beamer} % dvipdfmxなので必要

\usetheme{Boadilla}
%Beamerフォント設定
\usefonttheme{professionalfonts}
\usepackage[T1]{fontenc}
\usepackage[deluxe,uplatex]{otf} % 日本語多ウェイト化
\usepackage{mlmodern}  % 太いComputer Modern
% MLmodernのバグを修正: cf. https://tex.stackexchange.com/questions/646333/size-of-integral-symbol-in-section-header-with-mlmodern
\DeclareFontFamily{OMX}{mlmex}{}
\DeclareFontShape{OMX}{mlmex}{m}{n}{%
   <->mlmex10%
   }{}
\usepackage{newtxtext} % 数式以外をTXフォントで上書き
% \usepackage[noalphabet,unicode]{pxchfon}
% \setminchofont{KozMinPro-Regular.otf}% 游明朝Regular
% \setboldminchofont{KozMinPro-Bold.otf}% 游明朝Demibold
% \setgothicfont[0]{BIZ-UDGothicR.ttc}% BIZ UDゴシックR
% \setboldgothicfont[0]{BIZ-UDGothicB.ttc}% BIZ UDゴシックB
\renewcommand{\familydefault}{\sfdefault}  % 英文をサンセリフ体に
\renewcommand{\kanjifamilydefault}{\gtdefault}  % 日本語をゴシック体に
\usefonttheme{structurebold} % タイトル部を太字
\setbeamerfont{alerted text}{series=\bfseries} % Alertを太字
\setbeamerfont{section in toc}{series=\mdseries} % 目次は太字にしない
\setbeamerfont{frametitle}{size=\Large} % フレームタイトル文字サイズ
\setbeamerfont{title}{size=\LARGE} % タイトル文字サイズ
\setbeamerfont{date}{size=\small}  % 日付文字サイズ
\usepackage{bxcoloremoji}

% Babel (日本語の場合のみ・英語の場合は不要)
\uselanguage{japanese}
\languagepath{japanese}
\deftranslation[to=japanese]{Theorem}{定理}
\deftranslation[to=japanese]{Lemma}{補題}
\deftranslation[to=japanese]{Example}{例}
\deftranslation[to=japanese]{Examples}{例}
\deftranslation[to=japanese]{Definition}{定義}
\deftranslation[to=japanese]{Definitions}{定義}
\deftranslation[to=japanese]{Problem}{問題}
\deftranslation[to=japanese]{Solution}{解}
\deftranslation[to=japanese]{Fact}{事実}
\deftranslation[to=japanese]{Proof}{証明}
\def\proofname{証明}
\setbeamertemplate{theorems}[normal font] %日本語の場合,定理内を斜体にしない

%Beamer色設定
\definecolor{UniBlue}{RGB}{0,150,200}
\definecolor{UniGreen}{RGB}{3,175,122}
\definecolor{UniPink}{RGB}{255,128,255}
\definecolor{AlertOrange}{RGB}{255,76,0}
\definecolor{AlmostBlack}{RGB}{38,38,38}
\setbeamercolor{normal text}{fg=AlmostBlack}  % 本文カラー
\setbeamercolor{structure}{fg=UniBlue} % 見出しカラー
\setbeamercolor{block title}{fg=UniBlue!50!black} % ブロック部分タイトルカラー
\setbeamercolor{alerted text}{fg=AlertOrange} % \alert 文字カラー
\setbeamercolor{bibliography entry author}{fg=AlmostBlack} % Bib 文字カラー
\setbeamercolor{bibliography entry note}{fg=AlmostBlack} % Bib 文字カラー
\mode<beamer>{
    \definecolor{BackGroundGray}{RGB}{254,254,254}
    \setbeamercolor{background canvas}{bg=BackGroundGray} % スライドモードのみ背景をわずかにグレーにする
}

%フラットデザイン化
\setbeamertemplate{blocks}[rounded] % Blockの影を消す
\useinnertheme{circles} % 箇条書きをシンプルに
\setbeamertemplate{navigation symbols}{} % ナビゲーションシンボルを消す
\setbeamertemplate{footline}[frame number] % フッターはスライド番号のみ

%タイトルページ
\setbeamertemplate{title page}{%
    \vspace{2.5em}
    {\usebeamerfont{title} \usebeamercolor[fg]{title} \inserttitle \par}
    {\usebeamerfont{subtitle}\usebeamercolor[fg]{subtitle}\insertsubtitle \par}

    \vspace{1.5em}
    \begin{columns}[]
        \begin{column}{.35\linewidth}
        \usebeamerfont{titlegraphic}\inserttitlegraphic\par
        \end{column}
        \begin{column}{.6\linewidth}\setlength\topsep{0pt}
            \begin{flushright}
        \usebeamerfont{author}\insertauthor\par
        \usebeamerfont{institute}\insertinstitute \par
        \vspace{3em}
        \usebeamerfont{date}\insertdate\par
            \end{flushright}
        \end{column}
    \end{columns}
}

%biblatex
% \usepackage[style=ext-authoryear,citestyle=authoryear,natbib=true,giveninits=true,maxbibnames=99,maxcitenames=10,url=false,isbn=false,doi=false,articlein=false]{biblatex}
% \DeclareNameAlias{author}{first-last}
% \addbibresource{shrunk.bib}
% \newcommand{\mkbibbracketscol}[1]{\textcolor{UniBlue}{\scriptsize \mkbibbrackets{#1}}}
% \DeclareCiteCommand{\cite}[\mkbibbracketscol]
%   {\usebibmacro{prenote}}
%   {\usebibmacro{citeindex}%
%    \usebibmacro{cite}}
%   {\multicitedelim}
%   {\usebibmacro{postnote}}
% % \renewcommand{\cite}[1]{\textcolor{UniBlue}{\scriptsize\citep{#1}}}
% \renewcommand*{\finalnamedelim}{\addcomma\addspace} % remove "and" before the last author
% \DeclareDelimFormat{nameyeardelim}{\addspace}
% \setbeamertemplate{bibliography item}{}

% Algorithm系
\usepackage{algorithm}
\usepackage[noend]{algorithmic}
\algsetup{linenosize=\color{fg!50}\footnotesize}
\renewcommand\algorithmicdo{:}
\renewcommand\algorithmicthen{:}
\renewcommand\algorithmicrequire{\textbf{Input:}}
\renewcommand\algorithmicensure{\textbf{Output:}}

% TikZ
\usepackage{tikz}
\usetikzlibrary{positioning,shapes,arrows,quotes,shapes.callouts,calc,graphs,graphs.standard,fit,backgrounds,perspective}
\tikzset{point/.style={circle, fill=AlertOrange, inner sep=1pt, text width=1pt}}
\tikzset{bpoint/.style={circle, fill=black, inner sep=1pt, text width=1pt}}
\tikzset{alt/.code args={<#1>#2#3}{\alt<#1>{\pgfkeysalso{#2}}{\pgfkeysalso{#3}}}}
\tikzset{>=latex}
\tikzset{mynodes/.style={circle,white,fill=UniBlue!90,text width=.8em,inner sep=0pt,text centered,font=\footnotesize}}
\tikzset{myedges/.style={thick}}
\tikzset{mypath/.style={ultra thick, draw=AlertOrange}}
\tikzset{mypath2/.style={ultra thick, dashed, draw=UniGreen}}
\tikzset{mysubset/.style={draw,UniBlue,dotted,thick,fill=UniBlue!10}}
\tikzset{graphs/mygraph/.style={nodes={mynodes}, edges={myedges}, empty nodes}}
%% graph macros
\tikzgraphsset{declare={mybipartite}{%
subgraph I_nm [V={1,2,3,4,5,6}, W={a,b,c,d,e,f}];
1 -- {a,b};
2 -- {b,c,d,e};
3 -- {d,f};
4 -- {e,f};
{5,6} -- f;
}}
\tikzgraphsset{declare={mybipartitedi}{%
subgraph I_nm [V={1,2,3,4,5,6}, W={a,b,c,d,e,f}];
1 -> {a,b};
2 -> {b,c,d,e};
3 -> {d,f};
4 -> {e,f};
{5,6} -> f;
}}
\tikzgraphsset{declare={mybipartite2}{%
subgraph I_nm [V={1,2,3}, W={a,b,c}];
1 -- {a,b}; 2 -- {a,b,c}; 3 -- {a,c};
}} 

% Appendix
\usepackage{appendixnumberbeamer}

% macros
\newcommand{\R}{\mathbb{R}}
\newcommand{\Z}{\mathbb{Z}}
\newcommand{\be}{\mathbf{e}}
\newcommand{\ones}{\mathbf{1}}
% \newcommand{\bp}{\mathbf{p}}
\newcommand{\eps}{\varepsilon}
\newcommand{\defiff}{\overset{\text{def}}{\iff}}
\DeclareMathOperator{\poly}{poly}

\usepackage{mathtools}
\DeclarePairedDelimiter{\abs}{\lvert}{\rvert}
\DeclarePairedDelimiter{\norm}{\lVert}{\rVert}
\DeclarePairedDelimiter{\inprod}{\langle}{\rangle}
\usepackage[no-test-for-array]{nicematrix}
\usepackage{diagbox}

\newcommand{\highlightcap}[3][red]{\tikz[baseline=(x.base)]{\node[rectangle,rounded corners,fill=#1!10](x){#2} node[below=0pt of x, color=#1]{#3};}}
\newcommand{\highlight}[2][red]{\tikz[baseline=(x.base)]{\node[rectangle,rounded corners,fill=#1!10](x){#2};}}
\newcommand{\mycite}[2][\footnotesize]{\textcolor{UniBlue}{#1 [#2]}}

%section
% \AtBeginSection[]{
%     \frame[c]{
%         \centering
%         \begin{tikzpicture}
%             \node[white, circle, fill=UniBlue, text centered, text width=.3\linewidth](bullet) {
%                 {\Huge \bf \insertsectionnumber} \\[4em]
%     };
%     \node[text width=.65\linewidth, right=1em of bullet]{%
%         \usebeamerfont{title}\usebeamercolor[fg]{title}\insertsection\par %
%     };
%         \end{tikzpicture}
%     } %目次スライド
% }

%grid
% \usepackage[colorgrid,gridunit=pt,texcoord]{eso-pic}
\newcommand{\postit}[1]{\tikz[overlay, remember picture]{\node[draw,anchor=north east, align=center, rectangle, fill=yellow!10,xshift=-1pt,yshift=-1pt] at (current page.north east) {#1};}}

\usepackage{tcolorbox}
\newtcolorbox{primal}{colback=red!10!white, colframe=red, title=主問題 (P), left=2pt,right=2pt,top=0pt,bottom=0pt}
\newtcolorbox{dual}{colback=blue!10!white, colframe=blue, title=双対問題 (D), left=2pt,right=2pt,top=0pt,bottom=0pt}

% visible on
\usetikzlibrary{overlay-beamer-styles}